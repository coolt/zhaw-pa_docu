%%%%%%%%%%%%%%%%%%%%%%%%%%%%%%%%%%%%%%%%%%%%%%%%%%%%%%%%%%%%%%%%%
%  _____       ______   ____									%
% |_   _|     |  ____|/ ____|  Institute of Embedded Systems	%
%   | |  _ __ | |__  | (___    Wireless Group					%
%   | | | '_ \|  __|  \___ \   Zuercher Hochschule Winterthur	%
%  _| |_| | | | |____ ____) |  (University of Applied Sciences)	%
% |_____|_| |_|______|_____/   8401 Winterthur, Switzerland		%
%																%
%%%%%%%%%%%%%%%%%%%%%%%%%%%%%%%%%%%%%%%%%%%%%%%%%%%%%%%%%%%%%%%%%

\chapter{Diskussion und Ausblick}\label{chap.diskussion}

Das Provozieren von \textit{glitch} und von \textit{metastability} ist umgesetzt. Für die entwickelte MIDI-Schnittstelle fehlt die vollendete Integration des Blocks ins Synthesizer-Projekt. 

\section{Einbauen in das bestehende Synthesizer-Projekt}

Die Schnittstellen für die Integration sind im Anhang \ref{chap.anhang_top_synthesizer} definiert. Zwischen der MIDI-Schnittstelle und dem Audio-Codec wird ein Tone-Generator-Block eingebaut. Der Tone-Generator beinhaltet 3 Block-Einheiten, von denen zwei bereits im Synthesizer-Projekt aufgebaut sind: Einen Tone-Dekoder-Block (bestehend), einen FM-Synthesizer-Block (bestehend) und einen Mixer (neu). Letzterer muss aufgesetzt werden, da von den ersten zwei Block-Kategorien jeweils 10 Instanzen eingebaut sind (siehe Abbildung \ref{fig.top_synthesizer_detail}). 

Bereits umgebaut ist der Tone-Dekoder. Zur ursprünglichen Auswahl, dass 12 Töne über die Schalter eingestellt werden, werden neu 127 Notenwerte dekodiert. Der FPGA-Schalter 13 dient zur Auswahl, ob man die Töne vom Keyboard (Schalter auf logisch '1') oder wie bisher (bei Schalterstand logisch '0') von den restlichen 12 Schaltern auslösen will.\\
Bereits implementiert ist die 10-fache Instanzierung der Ton-Dekoder und die 10-fache Ausführung des bestehenden FM-Synthese-Blocks. Die Instanzen sind ins Top-Level eingebaut und verbunden (siehe Abbildung \ref{fig.rtl_top_synthesizer}). Die Ausgänge der 10 FM-Synthese-Blöcke führen auf einen Mixer. Von diesem besteht das Grundgerüst und ein erster Ansatz, die 10 Signale über Addition zu einem Signal zusammen zu führen. Dieser Ansatz ist nicht ausgereift und muss überprüft werden.

Da der Mixer die Signale nicht zu einem Signal vereint, konnte kein polyphones Signal an den Audio-Codec übertragen werden. Zur Zeit gibt das Syntthesizer-Projekt nut Töne über die Schalter aus.

\section{Frequenzmodulation mit vielfältiger Klangfarbe}

Was nicht implementiert ist, ist eine anspruchsvollere Frequenzmodulation mit vielfältiger Klangfarbe. Die bestehende Frequenzmodulation aus dem Synthesizer-Projekt instanziert zwei DDS, den einen als Carrier und der andere für die Modulation. Zusätzlich zu dieser Grundstruktur ist die Implementation von Seitenbändern angedacht (Konzept siehe \cite{synthesizer_1}, \cite{synthesizer_2}). Die bestehende Block-Struktur ermöglicht einen Ausbau.

\section{Refactoring Synzhesizer-Projekt}

Im bestehenden Synthesizer-Projekt sind gewisse Blocks direkt in das Top-Level eingebaut. Diese könnte man zu grösseren Einheiten gruppieren:
\begin{enumerate}
 \item Infrastructure (Für alle Clocks und IOs)
 \item MIDI-Schnittstelle (Bereits implementiert)
 \item Tone-Generator (Ansatzweise implementiert)
 \item Audio-Control (als Einheit über die vier VHDL-Audio-Blöcke)
\end{enumerate}

Beim Infrastructure-Block sind die Anschlüsse der Keys zu überarbeiten, da sie teilweise nicht angeschlossen sind und nicht über die Synchronisation führen. Alle Takte könnte man im Infrastructure-Block zentral generieren. Das bedeutet, dass auch die PLL und der Clock für den Audio-Codec (Signal BCK) in dem Infrastructure-Block integriert sind.

Einen professionelleren Eindruck könnten die Dateien machen, wenn die gleiche Konvention für Header-Angaben, Tab- oder Space-Einrückungen und für Kommentare angewendet werden. Generell stehen viele Kommentare in den Dateien und die Funktionsbeschreibungen sind knapp. Ein schnelleres Code-Verständnis könnten präzisere Namen der Blöcke bringen. Zur Zeit bestehen unterschiedliche Namens-Konventionen, was die Deutung der Funktion eines Blocks erschwert.

Ein Refactoring würde den Code verständlicher machen, und das Projekt als Ganzes aufwerten.

