%%%%%%%%%%%%%%%%%%%%%%%%%%%%%%%%%%%%%%%%%%%%%%%%%%%%%%%%%%%%%%%%%
%  _____       ______   ____									%
% |_   _|     |  ____|/ ____|  Institute of Embedded Systems	%
%   | |  _ __ | |__  | (___    Wireless Group					%
%   | | | '_ \|  __|  \___ \   Zuercher Hochschule Winterthur	%
%  _| |_| | | | |____ ____) |  (University of Applied Sciences)	%
% |_____|_| |_|______|_____/   8401 Winterthur, Switzerland		%
%																%
%%%%%%%%%%%%%%%%%%%%%%%%%%%%%%%%%%%%%%%%%%%%%%%%%%%%%%%%%%%%%%%%%

\chapter{Diskussion und Ausblick}\label{chap.diskussion}

Bespricht die erzielten Ergebnisse bezüglich ihrer ERwartbarkeit, Aussagekraft und Relevanz\\
Interpretation und Validierung der Resultate\\
Rückblick auf Aufgabenstellung: erreicht\ nicht erreicht\\

Legt dar, wie die Resultate weiterhin genutzt werden können\\ an sie angeschlossen werden kann\\

%----------------------------------------------------------------

Zwei unabhägige Projekte, unterschiedliche Hardware und Programme. Testbench braucht Zeit.

Ausstehend ist die Implementation in das bestehende Synthesizer-Projekt. Ein erster, schneller Versuch, die 10 Noten über 10 \textit{DDS} schnell auszugeben scheiterte, an der notwendigen Implementation eines Misches, der die 10 Noten zu einem Signal für den \textit{audio codec} zusammenfügt. 


Sehr schad, die vielfälltigen Klangfarben. Interessant. 






% Aus derr Einleitung------------------------------------------------------
Als offener Punkt besteht die Implementation des \textit{midi interfaces} in das bestehende Synthesizer-Projekt. Die Schnittstellen sind im Anhang festgehalten und die notewnigen Implementationsschritte, wie das Ausweiten des bestehenden DDS auf 10 DDS sind im Projekt als Blöcke eingebaut. Aus zeitlichen Gründen konnte dieser letzte Schritt nicht mehr während der Projektarbeit zu Ende gebracht werden. 