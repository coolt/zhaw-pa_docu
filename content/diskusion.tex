%%%%%%%%%%%%%%%%%%%%%%%%%%%%%%%%%%%%%%%%%%%%%%%%%%%%%%%%%%%%%%%%%
%  _____       ______   ____									%
% |_   _|     |  ____|/ ____|  Institute of Embedded Systems	%
%   | |  _ __ | |__  | (___    Wireless Group					%
%   | | | '_ \|  __|  \___ \   Zuercher Hochschule Winterthur	%
%  _| |_| | | | |____ ____) |  (University of Applied Sciences)	%
% |_____|_| |_|______|_____/   8401 Winterthur, Switzerland		%
%																%
%%%%%%%%%%%%%%%%%%%%%%%%%%%%%%%%%%%%%%%%%%%%%%%%%%%%%%%%%%%%%%%%%

\chapter{Diskussion und Ausblick}\label{chap.diskussion}

Das Provozieren von \textit{glitch} und von \textit{metastability} ist vollständig umgesetzt. Für das entwickelte \textit{midi interface} fehlt die Integration des Blocks ins Synthesizer-Projekt. 

\section{Einbauen in das bestehende Synthesizer-Projekt}

Ein erster Versuch, die 10 Noten über 10 \textit{DDS} auszugeben scheiterte aus zeitlichen Gründen. Das Anschliesen der 10 DDS ist gemacht, ebenso eine Testbench über das Top des Synthesizert. Die Umsetzng scheiterte an der Implementation ddes Mischers, der die 10 parallelen Noten zusammenfügt.

Im Anhang befindet sich ein Block-Schaltbild, wie das midi interface in das bestehende Projekt eingebaut wird. 

\section{Frequenzmodulation mit vielfälltiger Klangfarbe}
Artikel zu empfehlen


