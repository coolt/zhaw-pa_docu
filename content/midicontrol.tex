%%%%%%%%%%%%%%%%%%%%%%%%%%%%%%%%%%%%%%%%%%%%%%%%%%%%%%%%%%%%%%%%%
%  _____       ______   ____									%
% |_   _|     |  ____|/ ____|  Institute of Embedded Systems	%
%   | |  _ __ | |__  | (___    Wireless Group					%
%   | | | '_ \|  __|  \___ \   Zuercher Hochschule Winterthur	%
%  _| |_| | | | |____ ____) |  (University of Applied Sciences)	%
% |_____|_| |_|______|_____/   8401 Winterthur, Switzerland		%
%																%
%%%%%%%%%%%%%%%%%%%%%%%%%%%%%%%%%%%%%%%%%%%%%%%%%%%%%%%%%%%%%%%%%

\chapter{MIDI Steuerung}\label{chap.midi}

\section{Einleitung}\label{sect.einleitung_midi}
Im Modul DTP2 entwickeln Studentinnen und Studenten ihren eignenen Synthesizer. Eine Option in diesem Projekt ist es, den Synthesier über ein Keyboard per Midi steuern zu können. \\
Am Institut bestand bereits die MIDI UART und die Aufgabe im zweiten Teil dieser Projektarbeit ist es, die MIDI Steuerung zu entwickeln, die sowohl einezelne Töne weiterleitet wie auch polyphoniefähig ist (siehe nächstes Kapitel). \\

\begin{figure}[H]
	\centering
	\includegraphics[width=0.5\textwidth]{images/idle.png}
	\caption{blabla}
	\label{fig.bluetooth_}
\end{figure}


\section{Spezifikation Midi Protokoll}\label{sect.midi_spezification}



\section{Umsetzung Midi Interface}\label{sect.midi_umsetzung}
\subsection{Aufbau der Blöcke}
\subsection{Schnittstellen}
\subsection{Inhalt des zu entwickelnden Midi Controllers}
\subsection{Inhalt des zu entwickelnden Polyphonie Blocks}
