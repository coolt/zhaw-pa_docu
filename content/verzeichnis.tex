%%%%%%%%%%%%%%%%%%%%%%%%%%%%%%%%%%%%%%%%%%%%%%%%%%%%%%%%%%%%%%%%%
%  _____       ______   ____									%
% |_   _|     |  ____|/ ____|  Institute of Embedded Systems	%
%   | |  _ __ | |__  | (___    Wireless Group					%
%   | | | '_ \|  __|  \___ \   Zuercher Hochschule Winterthur	%
%  _| |_| | | | |____ ____) |  (University of Applied Sciences)	%
% |_____|_| |_|______|_____/   8401 Winterthur, Switzerland		%
%																%
%%%%%%%%%%%%%%%%%%%%%%%%%%%%%%%%%%%%%%%%%%%%%%%%%%%%%%%%%%%%%%%%%

\chapter{Verzeichnis}\label{chap.verzeichnis}


\bibliographystyle{plain}
\makeatletter
\renewcommand*\bib@heading{ \section{\refname}}
\makeatother

\bibliography{BibTex/references}

 


\section{Glossar}\label{sect.verzeichnis_glossar}
Das Glossar dient interessierten Software-Entwicklern, die elektrotechnik-spezifischen Worte zu verstehen.

\textbf{glitch}

tokens

\textbf{finate state machine}



\textbf{clock domain}


\textbf{textbasierte testbench}


\textbf{Durchlaufverzögerung}\\

Wird englisch \textit{propagation delay} genannt und bezeichnet die Zeit, die Daten vom Eingang bis zum Ausgang des Bauteils brauchen.\\
Die Durchlaufverzögerung beträgt beim Cylone IV 4 ns (Device Handbook, S. 8-19).\\


\textbf{hold time}\\
Ist die minimale Zeit, in der die Inputdaten \textit{nach} der Taktflanke stabil sein müssen.\\
Die hold-Zeit beträgt beim  Cyclone IV E 0 ns (Device Handbook, S. 8-19).\\


Pfadzeit\\
... (Unter 3.2. Metastabilität Ratschläge erwähnt)\\



\textbf{quartus}\\
IDE von altera zum Kompilieren, Synthesizieren und einbauen von IPs für die altera FPGAs.\\


\textbf{setup time} \\
minimale Zeit, in der Inputdaten stabil sein müssen be\textit{vor} ein Taktflanke die Daten triggert.\\
Die setup-Zeit beträgt beim Cyclone IV E 10 ns (Device Handbook, S. 8-19)\\




