%%%%%%%%%%%%%%%%%%%%%%%%%%%%%%%%%%%%%%%%%%%%%%%%%%%%%%%%%%%%%%%%%
%  _____       ______   ____									%
% |_   _|     |  ____|/ ____|  Institute of Embedded Systems	%
%   | |  _ __ | |__  | (___    Wireless Group					%
%   | | | '_ \|  __|  \___ \   Zuercher Hochschule Winterthur	%
%  _| |_| | | | |____ ____) |  (University of Applied Sciences)	%
% |_____|_| |_|______|_____/   8401 Winterthur, Switzerland		%
%																%
%%%%%%%%%%%%%%%%%%%%%%%%%%%%%%%%%%%%%%%%%%%%%%%%%%%%%%%%%%%%%%%%%

\chapter{Verzeichnis}\label{chap.verzeichnis}


\bibliographystyle{plain}
\makeatletter
\renewcommand*\bib@heading{ \section{\refname}}
\makeatother

\bibliography{BibTex/references}

 


\section{Glossar}\label{sect.verzeichnis_glossar}

Das Glossar dient interessierten Software-Entwicklern, die Elektrotechnik-spezifischen Worte zu verstehen.

\textbf{Asynchrone Signale}\\
Werden Signale unmittelbar zugewiesen sind sie vorerst ungetaktet, asynchron. Es ist nicht definiert, \textit{wann} exakt das Signal den neuen Wert erhält. Erst wenn ein Signal durch ein Flip-Flop geführt wird, wird es getaktet und seine Signalzuweisung dadurch determinierbar.

\textbf{Audio Codec}\\
Bezeichnet im vorgegebenen Synthesizer-Projekt den, bezüglich dem FPGA, externen Audio-Baustein auf dem Altera Development Board DE2-115. Es handelt sich um einen WM8731.

\textbf{Clock Domain}\\
Ein Bereich der Hardware, der mit demselben Takt läuft.

\textbf{Controller}\\
Bezeichnet ein Bauteil, das Eingangssignale gemäss einer Spezifikation verarbeitet und die entsprechenden Ausgangssignale setzt.

\textbf{DDS}\\
Bedeutet Direct Digital Synthesis und bezeichnet das digitale Erzeugen von periodischen Signalen. Diese Signale können für die Tonerzeugung gebraucht werden.

\textbf{Dekoder}\\
Bezeichnet ein Bauteil, das einen oder mehrere Eingangswert(e) gemäss implementierter Logik in einen Ausgangswert wandelt.

\textbf{Durchlaufverzögerung}\\
Wird englisch \textit{propagation delay} genannt und bezeichnet die Zeit, die Daten vom Eingang bis zum Ausgang des Bauteils brauchen.\\
Die Durchlaufverzögerung beträgt beim Cylone IV 4 ns \citep{Handbook_Altera}.

\textbf{Finite State Machine (fsm)}\\
''A model of a computational system, consisting of a set of states, a set of possible inputs, and a rule to map each state to another state, or to itself, for any of the possible inputs.'' \citep{fsm}\\
Auf deutsch: ''Ein Model in Rechensystemen, das aus einem Satz aus Zuständen, möglichen Eingängen und Regeln wie man von einem Zustand zum nächsten, oder zu sich selbst, für alle möglichen Eingänge gelangt.''

\textbf{Flip-Flops}\\
Grundbaustein der Digitalen Logik. Das Flip-Flop speichert seinen Wert, den es am Eingang erhält am Ausgang.

\textbf{Glitch}\\
Im technischem Bereich bedeutet \textit{glitch} gemäss Cambridge Dictionairy ''a sudden unexpected increase in electrical power, especially one that causes a fault in an electronic system'' \citep{dictionair}.\\
Auf deutsch: ''eine plötzliche, unerwartete Spannungserhöhung, die insbesondere ein Fehlverhalten im elektronischen System verursacht''.

\textbf{Hold Time}\\
Ist die minimale Zeit, in der die Inputdaten \textit{nach} der Taktflanke stabil sein müssen.\\
Die \textit{hold time} beträgt beim  Cyclone IV E 0 ns \citep{Handbook_Altera}.

\textbf{Hot Plug}\\
Bezieht sich auf die Hardware-Umsetzung einer \textit{finite state machine}. Gewöhnlich braucht es für \begin{math} 2^n \end{math} Zustände n Flip-Flops. Bei Hot Plug braucht es für n Zustände n Flip-Flops, denn jeder neue Zustand wird durch eine '1' am n-ten Flip-Flop detektiert. Alle anderen Flip-Flop-Werte sind auf '0'. Die logische Schaltung für eine \textit{Hot Plug fsm} wird durch den direkten Bezug einer gesetzten '1' zum Zustand einfach.


\textbf{Kathodenstrahl Osziloskop, KO}\\
Bezeichnet ein elektronisches Messgerät, das ein Signale analog als Spannungen mit deren zeitlichem Verlauf am Bildschirm ausgibt.

\textbf{Metastabilität}\\
Bezeichnet in der  digitalen Signalverarbeitung einen unsicheren Zustand. Der Wert des Ausgangssignals ist nicht vorhersehbar, da beim Eingangssignal die Daten zu spät ankommen oder zu früh weggenommen werden.

\textbf{Others}\\
Bezeichnet in einem Switch-Case in VHDL alle anderen Möglichkeiten, die nicht abgefragt werden. Es dient dem System einen definierten Zustand zu geben, falls etwas Unerwartetes eintrifft.

\textbf{Pfadzeit}\\
Bezeichnet die Zeit, die ein Signal von einem Flip-Flop zum nächsten braucht.

\textbf{Refactoring}\\
Bezeichnet das Überarbeiten eines funktionierenden Codes. Ziele sind, den Code effizienter, verständlicher und sicherer zu gestalten.

\textbf{Setup Time} \\
Minimale Zeit, in der Inputdaten stabil sein müssen \textit{bevor} eine Taktflanke die Daten triggert.\\
Die \textit{setup time} beträgt beim Cyclone IV E 10 ns \citep{Handbook_Altera}

\textbf{State}\\
Bezeichnet den aktuellen Zustand einer \textit{finite state machine}.

\textbf{Textbasierte Testbench}\\
In VHDL wird die Simulation der Signale in einer Testbench aufgesetzt. In der Testbench werden die Signalanregungen, Stimuli, definiert, und die zeitlichen Abläufe unter Signalen. Für eine Testbench ist eine eigene Software notwendig.\\
Eine textbasierte Testbench liest die Stimuli und die zu erwartenden Ergebnisse aus einer Datei ein. Die Ergebnisse werden ebenfalls in eine Datei ausgegeben.

\textbf{Token}\\
Bezeichnen Elemente in einer Reihe von strukturierten Daten.

\textbf{Quartus}\\
IDE von altera zum Kompilieren, Synthetisieren und Einbauen von IPs für die Altera FPGAs.
