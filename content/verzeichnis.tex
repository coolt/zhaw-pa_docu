%%%%%%%%%%%%%%%%%%%%%%%%%%%%%%%%%%%%%%%%%%%%%%%%%%%%%%%%%%%%%%%%%
%  _____       ______   ____									%
% |_   _|     |  ____|/ ____|  Institute of Embedded Systems	%
%   | |  _ __ | |__  | (___    Wireless Group					%
%   | | | '_ \|  __|  \___ \   Zuercher Hochschule Winterthur	%
%  _| |_| | | | |____ ____) |  (University of Applied Sciences)	%
% |_____|_| |_|______|_____/   8401 Winterthur, Switzerland		%
%																%
%%%%%%%%%%%%%%%%%%%%%%%%%%%%%%%%%%%%%%%%%%%%%%%%%%%%%%%%%%%%%%%%%

\chapter{Verzeichnis}\label{chap.verzeichnis}


\bibliographystyle{plain}
\makeatletter
\renewcommand*\bib@heading{ \section{\refname}}
\makeatother

\bibliography{BibTex/references}

 


\section{Glossar}\label{sect.verzeichnis_glossar}
Das Glossar dient interessierten Software-Entwicklern, die elektrotechnik-spezifischen Worte zu verstehen.

\textbf{Asynchrone Signale}\\
Alle nicht-getakteten Prozesse, da man bei ungetakteten Signalen nicht weiss, wann sie ihren Zustand ändern.

<<<<<<< HEAD
tokens
=======
\textbf{Dekoder}\\
>>>>>>> develop

\textbf{Flip-Flops}

\textbf{Glitch}\\
Im technischem Bereich bedeutet \textit{glitch} gemäss Cambridge Dictionaire ''a sudden unexpected increase in electrical power, especially one that causes a fault in an electronic system ''\cite{dictionair},\\
auf deutsch ''eine plötzliche, unerwartete Spannungserhöhung, die insbesondere ein Fehlverhalten im elektronischen System verursacht''.\\


\textbf{Finate State Machine}\\
''A model of a computational system, consisting of a set of states, a set of possible inputs, and a rule to map each state to another state, or to itself, for any of the possible inputs.''\cite{fsm}\\
Auf deutsch''Ein Model in Rechensystemen, das aus einem Satz aus Zuständen, möglichen Eingängen und Regeln wie man von einem Zustand zum nächsten, oder zu sich selbst, für alle möglichen Eingänge gelangt. ''\\


KO
Abkürzung für Kathondenstrahl Oszilloskop, bezeichnet die analoge Signalausgabe am Bildschirm.

\textbf{Clock Domain}


\textbf{Textbasierte Testbench}


\textbf{Durchlaufverzögerung}\\

Wird englisch \textit{propagation delay} genannt und bezeichnet die Zeit, die Daten vom Eingang bis zum Ausgang des Bauteils brauchen.\\
Die Durchlaufverzögerung beträgt beim Cylone IV 4 ns (Device Handbook, S. 8-19).\\


\textbf{Hold Time}\\
Ist die minimale Zeit, in der die Inputdaten \textit{nach} der Taktflanke stabil sein müssen.\\
Die hold-Zeit beträgt beim  Cyclone IV E 0 ns (Device Handbook, S. 8-19).\\


Pfadzeit\\
... (Unter 3.2. Metastabilität Ratschläge erwähnt)\\



\textbf{Quartus}\\
IDE von altera zum Kompilieren, Synthesizieren und einbauen von IPs für die altera FPGAs.\\


\textbf{Setup Time} \\
minimale Zeit, in der Inputdaten stabil sein müssen be\textit{vor} ein Taktflanke die Daten triggert.\\
Die setup-Zeit beträgt beim Cyclone IV E 10 ns (Device Handbook, S. 8-19)\\




