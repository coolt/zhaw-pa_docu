%%%%%%%%%%%%%%%%%%%%%%%%%%%%%%%%%%%%%%%%%%%%%%%%%%%%%%%%%%%%%%%%%
%  _____       ______   ____									%
% |_   _|     |  ____|/ ____|  Institute of Embedded Systems	%
%   | |  _ __ | |__  | (___    Wireless Group					%
%   | | | '_ \|  __|  \___ \   Zuercher Hochschule Winterthur	%
%  _| |_| | | | |____ ____) |  (University of Applied Sciences)	%
% |_____|_| |_|______|_____/   8401 Winterthur, Switzerland		%
%																%
%%%%%%%%%%%%%%%%%%%%%%%%%%%%%%%%%%%%%%%%%%%%%%%%%%%%%%%%%%%%%%%%%

\chapter{Einleitung}\label{chap.einleitung}

Die Projektarbeitet bietet die Möglichkeit, sich vertieft in VHDL einzuarbeiten. Mit vertieft ist das Erstellen eines eignen Projektes wie auch  das Kennenlernen der Eigenheiten der Sprache VHDL gemeint.\\

Der erste Teil der Projektarbeit umfasst die Auseinandersetzung mit der Sprache VHDL und deren eigenen Herausforderungen. Die zwei Fehlerquellen \textit{Glitches} und \textit{Metastabilität} werden künstlich erzeugt, damit ihre Ursache verstanden wird. \\

Der zweiten Teil der Projektarbeit beinhaltet eine Weiterentwicklung des Synthesizer-Projektes aud der Vorlesung Digitale Technik II. Konkret soll die Frequenzmodulation vertieft  und der Midianschluss implementiert werden.\\

Eine Projektarbeit verdient nicht ihren Namen, wenn am Schluss der Arbeit nicht ausführlich die Resultate diskutiert und die verwendete Literatur genannt wird. Es folgt deshalb ein ausführlicher Anhang, in dem unter anderem auch der verwendete VHDL-Code abgebildet ist.\\

------------------------------------------ALT \\


\section{Ausgangslage}\label{sect.einleitung_ausgangslage}
Nennt bestehende Arbeiten zu diesem Thema (Literaturrecherche)\\
Stand der Technik: Bisherige Lösungen des Problems und deren Grenzen\\
\\
\\
MIDI bedeutet \textit{musical instrument digital interface} und ist ein Standard, der sowohl die genaue Beschaffenheit der erforderlichen Hardware wie auch das Kommunikationsprotokoll der zu übermittelnden Daten festlegt (\textbf{Christian Braut: Das MIDI-Buch. Sybex, Düsseldorf u. a. 1993,}).\\
\textbf{The MIDI Manufacturers Association, MIDI 1.0 Detailed Specification, Document Version 4.2, Revised September 1995, Los Angeles (1995) }\\
\\
Im Modul DTP2 entwickeln Studentinnen und Studenten ihren eignenen Synthesizer. Eine Option in diesem Projekt ist es, den Synthesier über ein Keyboard per MIDI steuern zu können. \\
Am Institut for Embedded Systems bestand bereits die MIDI UART und die Aufgabe im zweiten Teil dieser Projektarbeit ist es, eine MIDI Steuerung zu entwickeln, die für das Modul DTP2 verwendet werden kann. Bild Top im Anhang\\
Am Institut for Embedded System bestand bereits die Hardware wie auch die MIDI UART. Diese detektiert die empfangenen Bytes und sendet ein valid-Flag, wenn das Byte korrekt ist. Das Byte wird als logic Vetor übermittelt.\\

Jeder zu entwickelnde Block wird mit einer textbasierten \textit{testbench} getestet. 

\section{Zielsetzung Aufgabenstellung Anforderungen}\label{sect.einleitung_ziele}
Formulierte Ziele der Arbeit\\
Verweist auf die \textbf{offizielle Aufgabenstellung des/der Dozenten im Anhang}\\
EV.Pflichtenheft, Spezifikation\\
EV. übersicht über die Arbeit: stellt die folgenden Teile kurz vor.\\


