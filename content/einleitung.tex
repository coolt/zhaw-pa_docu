%%%%%%%%%%%%%%%%%%%%%%%%%%%%%%%%%%%%%%%%%%%%%%%%%%%%%%%%%%%%%%%%%
%  _____       ______   ____									%
% |_   _|     |  ____|/ ____|  Institute of Embedded Systems	%
%   | |  _ __ | |__  | (___    Wireless Group					%
%   | | | '_ \|  __|  \___ \   Zuercher Hochschule Winterthur	%
%  _| |_| | | | |____ ____) |  (University of Applied Sciences)	%
% |_____|_| |_|______|_____/   8401 Winterthur, Switzerland		%
%																%
%%%%%%%%%%%%%%%%%%%%%%%%%%%%%%%%%%%%%%%%%%%%%%%%%%%%%%%%%%%%%%%%%

\chapter{Einleitung}\label{chap.einleitung}


\section{Ausgangslage}\label{sect.einleitung_ausgangslage}
Für den ersten Teil der Arbeit, die Timing Artifakte  \textit{glitch} und \textit{metastability} zu demonstrieren, gibt es wenige Referenzprojekte. Da die Zustände ungewollt sind, finden sie als Fehlerquellen Erwähnung in der Literatur\cite{ReferenceManual},  \cite{F_glitches}, \cite{F_metastability}. Nur ein Dokument ist gefunden, das die Erzeugung von \textit{ metastability} behandelt\cite{Metastabil}. Aus diesem Grund sich der Nachweis der Timing Artifakte auf Anregungen und der Erfahrung von Prof. Hans-Joachim Gelke.


Im zweiten Teil geht es um den Aufbau eines \textit{midi interfaces}. MIDI bedeutet \textit{musical instrument digital interface} und ist ein Standard, der die Beschaffenheit der Hardware wie auch das Kommunikationsprotokoll festlegt\cite{Midi_Braut}. Die MIDI Manufacturers Association dokumentiert die mehrfachen Erweitungen des MIDI 1.0 Standard\cite{Midi_specification}. Diese Spezifikationen bildet die Grundlage für den Block \textit{midi control}. Am Institut for Embedded Systems besteht ein \textit{midi uart top}-Block in VHDL von Armin Weiss. In dieser Projektarbeit zu entwickeln sind die zwei Einheiten \textit{midi control}  und \textit{polyphony out}. Und anschliessend diese Blocks in das bestehende Synthesizer-Projekt einzubauen. Bei beiden Blocks basiert die Entwicklung auf einer textbasierten \textit{testbench}.


Jeder zu entwickelnde Block wird mit einer textbasierten \textit{testbench} getestet. 

\section{Aufgabenstellung}\label{sect.einleitung_ziele}
Die offizielle Aufgabenstellung befindet sich im Anhang \ref{chap.anhang_aufgabenstellung}. 
 
\begin{itemize}
	\item Erzeugung von Glitches mit einem Zähler und nachgeschaltetem Dekoder. Sichtbarmachung der Glitches mit einem Oszilloskop. Betätigen des asynchronen Resets vom Decoder aus.

	\item Provozieren und sichtbarmachung von metastabilen Zuständen. Hierfür kann z.B. eine Schaltung mit zwei asynchronen externen Takten aufgebaut werden.
\end{itemize}  


Nach der Fertigstellen des ersten Teils, wird die Aufgabenstellung für den zweiten Teil präzisiert (siehe Anhang \ref{chap.anhang_aufgabenstellung_neu}).

\begin{itemize}
\item Midi Interface for Keyboard für Polyphonie nach Konzept von gelk
\begin{itemize}
    \item 10 Frequenz Control Ausgänge zur Steuerung der Tonhöhe des Generators
    \item 10 On/Off Ausgänge Ton on/off
    \item UART wird geliefert von gelk
    \item VHDL wird von Grund auf neu erstellt.
\end{itemize}
\item 10 DDS implementieren und mit Mischer mischen
\item Script basierte Testbench. Testbench erzeugt serielle Midi Daten, so wie sie auf dem DIN Stecker vorkommen (logisch)
\item Testbench liest eine Testscript Datei ein, in welcher die Tastendrücke eines Keyboards abgebildet werden können. Midi Poliphony Spec muss durch die Testbench unterstützt werden können. Velocity muss nicht unterstützt werden.
\item Kein VHDL code ohne Testbench.
\item Block level testbench. Unit Tests.
\end{itemize}
