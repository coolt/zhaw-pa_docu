%%%%%%%%%%%%%%%%%%%%%%%%%%%%%%%%%%%%%%%%%%%%%%%%%%%%%%%%%%%%%%%%%
%  _____       ______   ____									%
% |_   _|     |  ____|/ ____|  Institute of Embedded Systems	%
%   | |  _ __ | |__  | (___    Wireless Group					%
%   | | | '_ \|  __|  \___ \   Zuercher Hochschule Winterthur	%
%  _| |_| | | | |____ ____) |  (University of Applied Sciences)	%
% |_____|_| |_|______|_____/   8401 Winterthur, Switzerland		%
%																%
%%%%%%%%%%%%%%%%%%%%%%%%%%%%%%%%%%%%%%%%%%%%%%%%%%%%%%%%%%%%%%%%%

\chapter{Einleitung}\label{chap.einleitung}

Die Projektarbeitet bietet die Möglichkeit, sich vertieft in VHDL einzuarbeiten. Mit vertieft ist das Erstellen eines eignen Projektes gemeint, wie auch  das Kennenlernen der Spezialitäten dieser Sprache gemeint.
Der erste Teil der Arbeit kümmerte sich um die Auseinandersetzung mit der Sprache VHDL und deren eigenen Herausforderungen. Glitches und Metastabilität werden herbeigeführt, damit man ihr Auftreten verstanden wird. 
Im zweiten Teil wird das Projekt, aus der Vorlesung Digitale Technik II, weiterentwickelt.



Ziel: VHDL Programmierung vertiefen mit zwei \textbf{Grundübungen} zu Glitches und Metastabilität. Danach ein Synthesizer mit FPGA realsieren.\\
Motivation: Tiefer Einblick in VHDL-Coderiung und deren Fehler.\\
\newline

Vorschlag Vorgehen:\\
1. Glitches reporduzieren\\
2. Metastaibiltät bei Detektion nachweisen\\
