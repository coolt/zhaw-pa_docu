%%%%%%%%%%%%%%%%%%%%%%%%%%%%%%%%%%%%%%%%%%%%%%%%%%%%%%%%%%%%%%%%%
%  _____       ______   ____									%
% |_   _|     |  ____|/ ____|  Institute of Embedded Systems	%
%   | |  _ __ | |__  | (___    Wireless Group					%
%   | | | '_ \|  __|  \___ \   Zuercher Hochschule Winterthur	%
%  _| |_| | | | |____ ____) |  (University of Applied Sciences)	%
% |_____|_| |_|______|_____/   8401 Winterthur, Switzerland		%
%																%
%%%%%%%%%%%%%%%%%%%%%%%%%%%%%%%%%%%%%%%%%%%%%%%%%%%%%%%%%%%%%%%%%


\pagenumbering{Roman}

\appendix
\chapter{Anhang 1: Englische Definitionen Glitches}\label{chap.anhang_dictionaire}
%\begin{quote}A sudden, usually temporary malfunction or irregularity of equipment\end{quote},
 %oxford Dictionaries, (oxforddictionaries.com \textslach{de}\textslach{defintion}\textslach {englischusa}\textslach{glitch, 11.okt.2015)}\\
 
 
%\newline
%\begin{quote}A ​sudden ​unexpected ​increase in ​electrical ​power, ​especially one that ​causes a ​%fault in an ​electronic ​system\end{quote}, %http://dictionary.cambridge.org/de/worterbuch/englisch/glitch\\
%\newline

%\begin{quote}The \textit{glitch} is that unwanted \textit{spike} or transient output that increments some counter, clears some register, or starts some unwanted process at precisely the most undesirable time. This transient an undesirable spike is generally issued form any decoder that is addressed with a sequence of NON-UNIT-DISTANCE-CODED inputs.\end{quote}(Fletcher, Digital design, 472).\\





\newpage
\chapter{Anhang 2: VHDL-Code Glitch detect }\label{chap.anhang_2.vhdl_glitch}
%-------------------------------------------------------------------------------
%-- Project     : Glitches detect through long logic paths
%-- Description : glitch_detection.vhd    
%-- 				: Detect value 15. Once asynchronous (= glitch), once synchronous (cnt)         
%-- Author      : Katrin Bächli
%-------------------------------------------------------------------------------
%-- Change History
%-- Date     |Name      |Modification
%------------|----------|-------------------------------------------------------
%-- 05.10.15	| baek     | init
%-- 06.10.15 | baek     | add cnt-singal and clock
%-------------------------------------------------------------------------------
%
%
%library ieee;
%use ieee.std_logic_1164.all;
%use ieee.numeric_std.all;
%
%
%entity glitch_detection is
%	port(	clk: 				in std_logic;
%			glitch:			out std_logic; 
%			count:			out std_logic;	
%			-- Routing
%			q_0_out:			out std_logic;
%			q_1_out:			out std_logic;
%			q_2_out:			out std_logic;
%			q_3_out:			out std_logic;
%			------
%			q_0_in:			in  std_logic;
%			q_1_in:			in  std_logic;
%			q_2_in:			in  std_logic;
%			q_3_in:			in  std_logic
%	);
%end entity;
%
%
%
%architecture rtl of glitch_detection is 
%
%----------------------------------------------------------------------------------
%-- signal
%----------------------------------------------------------------------------------
%
%signal  cnt_async: 		integer range 0 to 15 		  := 0;
%signal  next_cnt_async: integer range 0 to 15 		  := 0;
%signal  detect_15_async: std_logic 						  := '0';  
%
%signal  cnt_sync:			std_logic						  := '0';
%signal  cnt_sync_next:	std_logic						  := '0';  
%
%
%signal  rout_out:       std_logic_vector(7 downto 0) := "00000000";
%signal  rout_in:        std_logic_vector(7 downto 0) := "00000000";
%
%
%
%begin
%
%------------------------------------------------------
%-- input
%------------------------------------------------------	
%		
%	count_up: process(ALL)	
%	begin
%		next_cnt_async <= cnt_async + 1;
%	end process;
%
%	
%------------------------------------------------------
%-- clocked processes
%------------------------------------------------------	
%	ff: process(clk)	
%	begin			
%		if (rising_edge(clk)) then		
%				cnt_async 		<= next_cnt_async;
%				cnt_sync 		<= cnt_sync_next;					
%		end if;
%	end process;
%
%	
%------------------------------------------------------
%-- delay
%------------------------------------------------------
% 
%	rout_out <= std_logic_vector(to_unsigned(cnt_async, 8));
%   q_0_out  <=  rout_out(0);   
%   q_1_out  <=  rout_out(1);
%   q_2_out  <=  rout_out(2);
%   q_3_out  <=  rout_out(3);  
%   ------------------
%	rout_in(0)	  <=  q_0_in;   
%   rout_in(1)	  <=  q_1_in;
%   rout_in(2)	  <=  q_2_in;		
%		
%		
%------------------------------------------------------
%-- output
%------------------------------------------------------	
%	
%	reset_counter: process(ALL)	
%	begin	
%	   -- asynchronous
%		if ( rout_in(0) = '1' AND rout_in(1) = '1' AND rout_out(2) = '1' AND rout_out(3) = '1') then  
%			detect_15_async <= '1';
%			cnt_sync_next <= '1';
%		else 
%			detect_15_async <= '0';
%			cnt_sync_next <= '0';
%		end if;	
%	end process;
%		
%	-- set outputs
%	count  <= cnt_sync;
%	glitch <= detect_15_async;
%
%end rtl;

\section{Bliblibli}\label{sect.anhang.ah}



\begin{itemize}
	\item sfasdfasdfasfasf
\end{itemize}
\newpage
