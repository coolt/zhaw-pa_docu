%%%%%%%%%%%%%%%%%%%%%%%%%%%%%%%%%%%%%%%%%%%%%%%%%%%%%%%%%%%%%%%%%
%  _____       ______   ____									%
% |_   _|     |  ____|/ ____|  Institute of Embedded Systems	%
%   | |  _ __ | |__  | (___    Wireless Group					%
%   | | | '_ \|  __|  \___ \   Zuercher Hochschule Winterthur	%
%  _| |_| | | | |____ ____) |  (University of Applied Sciences)	%
% |_____|_| |_|______|_____/   8401 Winterthur, Switzerland		%
%																%
%%%%%%%%%%%%%%%%%%%%%%%%%%%%%%%%%%%%%%%%%%%%%%%%%%%%%%%%%%%%%%%%%

\chapter*{Vorwort}\label{chap.vorwort}
an Alexey: Bitte hier auf inhaltliche Richtigkeit prüfen. Brauche es erst am Do
korrekt ...

Meine Motivation ist das vertiefte Kennenlernen der Sprache VHDL. Diese hardwarenahe Sprache beinhaltet mit der kombinatorischen Logik und der auch nicht sequentiellen Prozessverarbeitung Eigenheiten, mit denen ich Umgehen lernen wollte.

In der Projektarbeit waren es exakt diese Punkte, mit denen ich viel Zeit durch Debuggen verbrachte. Doch gerade so, ist mir nun diese Art der Programmierung vertrauter geworden und ich freue mich, auf kommende VHDL-Projekte.

Ich möchte Prof. Hans-Joachim Gelke meinen Dank aussprechen. Er legte den Fokus immer wieder auf die kombinatorische Logik und die Konsequenz des Codes, für das Umsetzen in der Hardware. Ebenfalls möchte ich Dr. Matthias Rosenthal danken, der diskret im Hintergrund die Arbeit mittrug und den Entwicklungsprozess mittrug.

Ich denke, dass diese Arbeit vor allem für Software Ingenieure interessant ist, da sie einen groben Einblick in die hardwarenahe Programmierung erhalten.



