%%%%%%%%%%%%%%%%%%%%%%%%%%%%%%%%%%%%%%%%%%%%%%%%%%%%%%%%%%%%%%%%%
%  _____       ______   ____									%
% |_   _|     |  ____|/ ____|  Institute of Embedded Systems	%
%   | |  _ __ | |__  | (___    Wireless Group					%
%   | | | '_ \|  __|  \___ \   Zuercher Hochschule Winterthur	%
%  _| |_| | | | |____ ____) |  (University of Applied Sciences)	%
% |_____|_| |_|______|_____/   8401 Winterthur, Switzerland		%
%																%
%%%%%%%%%%%%%%%%%%%%%%%%%%%%%%%%%%%%%%%%%%%%%%%%%%%%%%%%%%%%%%%%%

\chapter*{Vorwort}\label{chap.vorwort}

Meine Motivation ist das vertiefte Kennenlernen der Sprache VHDL. Diese hardwarenahe Sprache beinhaltet mit der kombinatorischen Logik und der auch nicht-sequentiellen Prozessverarbeitung Eigenheiten, mit denen ich vertraut werden will. 

Der erste Teil der Projektarbeit, das Provozieren von Signalfehlern, lässt mich in die asynchrone Signalverarbeitung einblicken und wird meinen VHDL-Coderstil nachhaltig prägen. Im zweiten Teil, dem Entwickeln eines \textit{midi interfaces} lerne ich ein Protokoll zu durchleuchten. Besonders interessant ist die textbasierte \textit{testbench}, welche die Implementation auf Herz und Nieren testet.

Ich möchte Prof. Hans-Joachim Gelke Dank aussprechen. Er lehrt mich viel über kombinatorische Logik. Ebenfalls möchte ich Dr. Matthias Rosenthal danken, der die Arbeit und den Entwicklungsprozess mitträgt.

Aus meiner Sicht ist diese Arbeit vor allem für Software Ingenieure interessant, da sie einen  Einblick in die hardwarenahe Programmierung gibt.

Ich freue mich auf kommende VHDL-Projekte.
