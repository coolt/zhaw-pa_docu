%%%%%%%%%%%%%%%%%%%%%%%%%%%%%%%%%%%%%%%%%%%%%%%%%%%%%%%%%%%%%%%%%
%  _____       ______   ____									%
% |_   _|     |  ____|/ ____|  Institute of Embedded Systems	%
%   | |  _ __ | |__  | (___    Wireless Group					%
%   | | | '_ \|  __|  \___ \   Zuercher Hochschule Winterthur	%
%  _| |_| | | | |____ ____) |  (University of Applied Sciences)	%
% |_____|_| |_|______|_____/   8401 Winterthur, Switzerland		%
%																%
%%%%%%%%%%%%%%%%%%%%%%%%%%%%%%%%%%%%%%%%%%%%%%%%%%%%%%%%%%%%%%%%%

\chapter{Resultate der Projektarbeit}\label{chap.resultate}

Zusammenfassung der Resultate

\section{Generieren von Glitches}\label{sect.resultateGlitch}
Beides erreicht. Viel Aufwand, da wenig Wissen wie ungewollter Zustand erzeugt werden kann.
Es dauerte 4 Wochen (15. oktober + Doku), der insgesammt 16 Wochen PA.

\section{Zustand von Metastabilität provozieren}\label{sect.resultateMetastabilty}
Beides erreicht. Viel Aufwand, da wenig Wissen wie ungewollter Zustand erzeugt werden kann.
Es dauerte 4 Wochen (15. oktober + Doku), der insgesammt 16 Wochen PA.

\section{MIDI Controller entwickeln}\label{sect.resultateMidiControl}

\section{Polyphonie Block }\label{sect.resultatePolyphonie}
Midiansteuerung nach vielen Redesignes gelungen.
Mehr in der Software geübt, ist das Timing in VHDL übungsbedürftig.

\section{DDS Generatoren basierend auf Frequenzmodulation entwickeln}\label{sect.resultateFrequenzmodulation}
Sitzung: Nur 10 DDS einbauen.
Schnittstellen da.
Nicht da,Frequenzmodulation anstelle von LUT. Aus zeitgründen.
Nur erster Entwurf, wie es umzusetzen ist.
\section{Textbasierte Testbench für alle entwickelten Blocks}\label{sect.resultateTestbench}
 




