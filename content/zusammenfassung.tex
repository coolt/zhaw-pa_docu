%%%%%%%%%%%%%%%%%%%%%%%%%%%%%%%%%%%%%%%%%%%%%%%%%%%%%%%%%%%%%%%%%
%  _____       ______   ____									%
% |_   _|     |  ____|/ ____|  Institute of Embedded Systems	%
%   | |  _ __ | |__  | (___    Wireless Group					%
%   | | | '_ \|  __|  \___ \   Zuercher Hochschule Winterthur	%
%  _| |_| | | | |____ ____) |  (University of Applied Sciences)	%
% |_____|_| |_|______|_____/   8401 Winterthur, Switzerland		%
%																%
%%%%%%%%%%%%%%%%%%%%%%%%%%%%%%%%%%%%%%%%%%%%%%%%%%%%%%%%%%%%%%%%%

\chapter*{Zusammenfassung}

Der erste Teil der Projektarbeit befasst sich mit zwei Herausforderungen der hardwarenahen Sprache VHDL. 

Als Erstes werden sogenannte \textit{glitches}, ungewollte Signalspitzen, künstlich herbeigeführt. Dies geschieht, in dem man die Signalpfade über externes Routing verlängert und dadurch der logische Wert verzögert beim nächsten Bautei, einem \textit{decoder}, eintrifft. Wird der \textit{decoder} asynchron betrieben, so verarbeitet dieser kurzzeitig falsche Werte, was in einem \textit{glitch} sichtbar gemacht wird. 

Als Zweites wird in einer Schaltung ein metastabiler Zustand provoziert. Damit dies erreicht wird, muss die \textit{hold} - oder die\textit{ setup time} eines Flip-Flops verletzt werden. Erzeugt wird die Metastabilität durch unterschiedliches Takten zweier Logiken. Der Ausgang der ersten \textit{clock domain} wird als asynchron Inpuls auf eine \textit{finate state machine} einer anderen \textit{clock domain} geführt. Die \textit{finate state machine} fällt nach kürzester Zeit in einen undefinierten Zustand, einen Zustand, den sie nicht implementiert hat.



Der zweiten Teil der Projektarbeit beinhaltet das Entwickeln eines polyphonen \textit{midi interface} für das Synthesizer-Projektes der Vorlesung Digitaltechnik II. Gemäss dem MIDI 1.0 Standard wird ein Controller implementiert, der auch das Drücken mehrer Tasten zuverlässig detektiert. Um die Entwicklung effizient zu gestalten, wird von Beginn weg mit einer textbasierten \textit{testbench} gearbeitet. 
Für die Ausgabe der maxibal 10 gleichzeitig ertönenden Noten wird ein zweiter Block für das Händeln der ein und aus der einzelnen NOten geschrieben. Beide Blocks sind eingehend mit der testbench getestet und deren Verhalten gut dokumentiert.

(weglassen ??? Erst am Schluss nennen.... )
Als offener Punkt besteht die Implementation des \textit{midi interfaces} in das bestehende Synthesizer-Projekt. Die Schnittstellen sind im Anhang festgehalten und die notewnigen Implementationsschritte, wie das Ausweiten des bestehenden DDS auf 10 DDS sind im Projekt als Blöcke eingebaut. Aus zeitlichen Gründen konnte dieser letzte Schritt nicht mehr während der Projektarbeit zu Ende gebracht werden. 


