%%%%%%%%%%%%%%%%%%%%%%%%%%%%%%%%%%%%%%%%%%%%%%%%%%%%%%%%%%%%%%%%%
%  _____       ______   ____									%
% |_   _|     |  ____|/ ____|  Institute of Embedded Systems	%
%   | |  _ __ | |__  | (___    Wireless Group					%
%   | | | '_ \|  __|  \___ \   Zuercher Hochschule Winterthur	%
%  _| |_| | | | |____ ____) |  (University of Applied Sciences)	%
% |_____|_| |_|______|_____/   8401 Winterthur, Switzerland		%
%																%
%%%%%%%%%%%%%%%%%%%%%%%%%%%%%%%%%%%%%%%%%%%%%%%%%%%%%%%%%%%%%%%%%

\chapter*{Zusammenfassung}


Die hardwarenahe Programmiersprache VHDL ist in der heutigen Zeit ein wichtiger Bestandteil der digitale Signalverarbeitung.Die Projektarbeit befasst sich mit zwei Themen von VHDL:\\
\begin{itemize}
\item Das Erforschen zweier hardwarenahen Fehlerquellen, \textit{glitches} und \textit{metastability}, und 
	\item das Entwickeln eines \textit{midi interfaces}, bei dem die Entwicklung von einer \textit{texbasierten testbench} begleitet wird.
\end{itemize} 


Im Fokus der VHDL-Arbeit steht das künstlich Herbeiführen von \textit{glitches}. Dies gelingt mit einem Cylcone II-FPGA in dem Pfade einzelner Signale künstlich verlängert sind. Dadurch treffen Werte verzögert ein und der asynchrone \textit{decoder} verarbeitet falsche Werte. Es gelangen falsche Signale auf die Leitung, sogenannte \textit{glitch}. 


Um einen metastabilen Zustand in einem System zu provozieren, sind zwei VHDL-Logik-Blöcke unterschiedlich getaktet. Kein Takt wird als Vielfaches des anderen implementiert. Das Ausgangssignal des ersten Logik-Blocks wird als asynchron Inpuls auf den zweiten Logik-Block geführt. Dekodiert die \textit{finate state machine} keinen definierten Zustände, ist das System in einen undefinierten Zustand gefallen. Metastabilität trifft ein.\\


Der zweiten Teil der Projektarbeit entwickelt \textit{midi interface}, das Polyphonie detektiert. Die textbasierte \textit{testbench} begleitet die Entwicklung des \textit{midi controller}. Das \textit{midi interface} detektiert die \textit{status bytes} NOTE ON, NOTE OFF und POLYPHONY und die VHDL-Einheit \textit{polyphony} out gibt 10 gedrückte Noten parallel aus.\\







