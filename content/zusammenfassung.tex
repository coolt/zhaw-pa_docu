%%%%%%%%%%%%%%%%%%%%%%%%%%%%%%%%%%%%%%%%%%%%%%%%%%%%%%%%%%%%%%%%%
%  _____       ______   ____									%
% |_   _|     |  ____|/ ____|  Institute of Embedded Systems	%
%   | |  _ __ | |__  | (___    Wireless Group					%
%   | | | '_ \|  __|  \___ \   Zuercher Hochschule Winterthur	%
%  _| |_| | | | |____ ____) |  (University of Applied Sciences)	%
% |_____|_| |_|______|_____/   8401 Winterthur, Switzerland		%
%																%
%%%%%%%%%%%%%%%%%%%%%%%%%%%%%%%%%%%%%%%%%%%%%%%%%%%%%%%%%%%%%%%%%

\chapter*{Zusammenfassung}

Die hardwarenahe Programmiersprache VHDL ist in der heutigen Zeit ein wichtiger Bestandteil der digitale Signalverarbeitung. Bestandteil dieser Projektarbeit sind zwei VHDL-Themen: Das Erforschen zweier hardwarenahen Fehlerquellen, \textit{glitches} und \textit{metastability}, und ein über \textit{texbasierte testbench} zu entwickelndes \textit{midi interface}.\\

Im Fokus der \textit{glitches} steht das Provozieren der ungewollten Signalspitze. Dies gelingt, in dem man in VHDL mit einem Cylcone II einzelne Signalpfade künstlich verlängert: Das Eintreffen dieser Signale verzögert sich. Dadurch entschlüsselt der asynchrone Dekoder kurzzeitig falsche Werte, die als \textit{glitches} auf die Signalleitung gelangen.\\

Zum Provozieren eines metastabilen Zustandes wird in VHDL ein System mit zwei \textit{clock domains} aufgebaut. Kein Clock ist ein Vielfaches des anderen. In eine \textit{clock domain} sendet Impulse an die andere. Diese verarbeitet diese in einer \textit{finate state machine}. Die aktuellen Zustände der \textit{fsm} werden an LEDs ausgegeben. Tritt keiner der vordefinierten Zustände ein, ist das System in ein undefinierten Zustand gefallen. Metastabilität tritt ein.\\

Das zweite VHDL-Thema ist die Entwickeln eines \textit{midi interface} über eine \textit{texbasierte testbench}. Bevor einzelne Einheiten des \textit{midi interfaces} in VHDL geschrieben werden, wird die \textit{testbench} erstellt. Die \textit{testbench} enthält Fehlerquellen gemäss MIDI 1.0 Spezifikation. Ziel der VHDL-Codierung ist, einen Fehler nach dem anderen zu lösen. \\

Das \textit{midi interface} wird für einen Cyclone IV kompiliert. Die MIDI \textit{status bytes} NOTE ON, NOTE OFF und POLYPHONY sollen detektiert und die \textit{data bytes} gemäss Spezifikation verarbeitet werden. Nach der Dekodierung folgt eine VHDL-Einheit, die die 10 gedrückten MIDI-Noten parallel mit inklusive ihrem Zustand ON oder OFF ausgibt. 





