%%%%%%%%%%%%%%%%%%%%%%%%%%%%%%%%%%%%%%%%%%%%%%%%%%%%%%%%%%%%%%%%%
%  _____       ______   ____									%
% |_   _|     |  ____|/ ____|  Institute of Embedded Systems	%
%   | |  _ __ | |__  | (___    Wireless Group					%
%   | | | '_ \|  __|  \___ \   Zuercher Hochschule Winterthur	%
%  _| |_| | | | |____ ____) |  (University of Applied Sciences)	%
% |_____|_| |_|______|_____/   8401 Winterthur, Switzerland		%
%																%
%%%%%%%%%%%%%%%%%%%%%%%%%%%%%%%%%%%%%%%%%%%%%%%%%%%%%%%%%%%%%%%%%

\chapter*{Zusammenfassung}


Die hardwarenahe Programmiersprache VHDL ist ein wichtiger Bestandteil der digitalen Signalverarbeitung. Die Projektarbeit setzt zwei unabhängige Aufgaben in VHDL um.

\begin{itemize}
\item Bilden zweier hardwarenahen Fehlerquellen: \textit{Glitches} und \textit{Metastability}
	\item Implementation eines \textit{MIDI Interfaces}, dessen Entwicklung auf einer \textit{textbasierten Test Bench} basiert.
\end{itemize} 

Durch das Verlängern der Pfade einzelner Signale auf einem Cyclone II-FPGA werden \textit{Glitches} künstlich herbeigeführt. Da einzelne Werte verzögert eintreffen, verarbeitet der asynchrone \textit{Decoder} falsche Werte. Unerwünschte Signale gelangen auf die Leitung und sogenannte Glitches entstehen. 

Der metastabile Zustand in einem System entsteht durch das Verletzen der Setup und Hold Time. Forciert ist dies durch zwei unterschiedliche Takte zweier VHDL-Logik-Blöcke, deren Takt-Phasen nicht übereinstimmen. Im Code ist das Ausgangssignal des ersten Logik-Blocks als asynchroner Impuls auf den zweiten Logik-Block geführt, der eine \textit{Finite State Machine} enthält. Dekodiert die FSM keinen definierten Zustand, befindet sich das System in einem undefinierten Zustand. Metastabilität trifft ein.

Der zweite Teil der Projektarbeit beinhaltet ein \textit{MIDI Interface}, das Polyphonie detektiert. Die textbasierte \textit{Test Bench} begleitet die Entwicklung des \textit{MIDI Controller}. Das \textit{MIDI Interface} detektiert die \textit{Status Bytes} NOTE ON, NOTE OFF und POLYPHONY und die VHDL-Einheit \textit{Polyphony Out} gibt 10 gedrückte Noten parallel aus.
