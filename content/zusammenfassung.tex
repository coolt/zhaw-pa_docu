%%%%%%%%%%%%%%%%%%%%%%%%%%%%%%%%%%%%%%%%%%%%%%%%%%%%%%%%%%%%%%%%%
%  _____       ______   ____									%
% |_   _|     |  ____|/ ____|  Institute of Embedded Systems	%
%   | |  _ __ | |__  | (___    Wireless Group					%
%   | | | '_ \|  __|  \___ \   Zuercher Hochschule Winterthur	%
%  _| |_| | | | |____ ____) |  (University of Applied Sciences)	%
% |_____|_| |_|______|_____/   8401 Winterthur, Switzerland		%
%																%
%%%%%%%%%%%%%%%%%%%%%%%%%%%%%%%%%%%%%%%%%%%%%%%%%%%%%%%%%%%%%%%%%

\chapter*{Zusammenfassung}


Die hardwarenahe Programmiersprache VHDL ist ein wichtiger Bestandteil der digitale Signalverarbeitung. Die Projektarbeit setzt zwei unabhängige Aufgaben in VHDL um.

\begin{itemize}
\item Bilden zweier hardwarenahen Fehlerquellen: \textit{glitches} und \textit{metastability}
	\item Implementation eines \textit{midi interfaces}, dessen Entwicklung auf einer \textit{texbasierten testbench} basiert
\end{itemize} 

 \textit{Glitches} werden künstlich Herbeigeführt in dem auf einem Cylcone II-FPGA die Pfade einzelner Signale verlängert sind. Dadurch treffen Werte verzögert ein und der asynchrone \textit{decoder} verarbeitet falsche Werte. Falsche Signale gelangen auf die Leitung und sogenannte \textit{glitch} entstehen. 

Der metastabilen Zustand in einem System entsteht, durch das Verletzen der \textit{setup} und \textit{hold time}. Forciert ist dies durch zwei unterschiedliche Takte zweier VHDL-Logik-Blöcke, deren Takt-Phasen nicht übereinstimmen. Das Ausgangssignal des ersten Logik-Blocks ist als asynchron Inpuls auf den zweiten Logik-Block geführt, der eine \textit{finate state machine} enthält. Dekodiert die \textit{fsm} keinen definierten Zustände, befindet sich das System in einen undefinierten Zustand. Metastabilität trifft ein.

Der zweiten Teil der Projektarbeit beinhaltet ein \textit{midi interface}, das Polyphonie detektiert. Die textbasierte \textit{testbench} begleitet die Entwicklung des \textit{midi controller}. Das \textit{midi interface} detektiert die \textit{status bytes} NOTE ON, NOTE OFF und POLYPHONY und die VHDL-Einheit \textit{polyphony} out gibt 10 gedrückte Noten parallel aus.
